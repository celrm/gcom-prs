\documentclass{article}

\author{Celia Rubio Madrigal}
\title{Práctica 1 - GCOMP}
\date{9 de febrero de 2022}

\begin{document}
	\maketitle
	
	Introducción - motivación
	qué queremos obtener
	analizar la convergencia de un sistema dinámico no lineal llamado logística que representa especies 2 3 lineas
	
	Material usado
	datos que nos den, plantilla, escrito en ingles a codificar, etc
	conjunto de condiciones iniciales parametros etc
	datos de entrada
	set de datos
	condiciones
	contorno espaciales temporales espaciales fijados
	
	
	metodologia
	implementacion del sistema din no lineal mediante iteraciones y deteccion de cotas de error explicar calcular cota de error tomando conjunto de datos candidatos a la cuenca de atraccion menos la cuenca anterior, poner formulita
	
	resultados y discusion (a veces junto cuando los resultados son pocos)

	datos e interpretacion.
	para r=3,2 obtenfo esto, para tal obtengo cual
	
	interpretable. obtengo que los tiempos de transicicon eran pequeños en comparacion con la computacion... que la estimacion de cota claramente es muy inferior a la diferencia de puntos que estoy obteniendo
	
	conclusiones (lo mismo...) no hace falta 10 apartados, pero el contenido tiene que estar
	
	graficas visuales. cuantas? el 50 como maximo
	
	criterios graficas vectoriales. memoria en latex jeje si pesan mucho esta bien, si tiene mala resolucion pero se ve no pasa nada
	
	el indice no cuenta, anexos tampoco. uno del codigo. y adjuntar el codigo independiente.	
	
	unir apartados segun sea necesario
	por ej metodos y datos
	
\end{document}